\documentclass{llncs}
\usepackage[utf8]{inputenc}
\usepackage[T1]{fontenc}
\usepackage[scaled=0.8]{beramono}
\usepackage{amsmath}
\usepackage{amssymb}
\usepackage{stmaryrd}
\usepackage{mathtools}
\usepackage{tabto}
\usepackage{comment}
\usepackage{xspace}
\usepackage{todonotes}

\usepackage{hyperref}
\hypersetup{allcolors=blue,colorlinks=true}

\usepackage{cleveref}
% \usepackage[numbers,sort&compress]{natbib}
\newcommand{\mailto}[1]{\href{mailto:#1}{\ttfamily #1}}

\newcommand{\code}[1]{\texttt{#1}}

\newcommand{\Legion}{\textsc{Legion}\xspace}
\newcommand{\SymCC}{\textsc{SymCC}\xspace}
\newcommand{\Angr}{\texttt{angr}\xspace}

\author{
    Gidon Ernst
        \inst{1} \thanks{Jury Member}
\and
    Dongge Liu
        \inst{2}
\and
    Toby Murray
        \inst{2}
\and
    Thuan Pham
        \inst{2}
}

\title{\Legion/\SymCC---Tool Description \\ (Competition Contribution)}

\institute{
    LMU Munich, \mailto{gidon.ernst@lmu.de}
\and
    University of Melbourne,
    \mailto{donggel@student.unimelb.edu.au},
    \mailto{toby.murray@unimelb.edu.au},
    \mailto{thuan.pham@unimelb.edu.au}
}
\pagestyle{plain}

\begin{document}

\maketitle

\begin{abstract}

\end{abstract}
    \keywords{Fuzz Testing \and Concolic Execution}

\section{Testing Approach}
\label{sec:approach}

\Legion/\SymCC implements a testing strategy similar to \Legion:
The program is explored through concrete executions where the nondeterministic input values are in part precomputed and in part randomly generated.
High coverage is achieved by making informed decisions about which parts of the program should be executed next.
To that end, both versions of \Legion maintain a \emph{tree model} of execution traces, where internal nodes correspond to branching points in the program's machine code.
Nodes in the tree are annotated by the respective symbolic path constraints.
\Legion's main loop consists of four stages:
select a node from the tree accoring to a \emph{score function},
solve its path constraint to obtain an input prefix that triggers that node during execution,
run the binary such that all further nondeterministic choices are done randomly,
and finally collect a trace of the execution and integrate it into the tree.
\Legion therefore leverages \emph{symbolic constraint solving} to pass through
narrow choke points in the program (e.g. magic numbers),
and \emph{random fuzzing} to cheaply cover those parts of the program that are easy to target.


\section{Software Architecture}
\label{sec:architecture}

\Legion/\SymCC started out as an attempt to implement the original \Legion algorithm 
using \SymCC as a backend instead of \Angr.
\Legion/\SymCC is in large parts a re-implementation, because \SymCC works fundamentally different to \Angr:
Binary execution and generation of symbolic constraints is merged in a single step, which greatly simplifies the design and avoids mismatches between the two.
As a consequence, \SymCC is much faster.

The high-level algorithm of \Legion/\SymCC is implemented as a Python program.
The binary itself is instrumented by the \SymCC compiler pass and a custom runtime
that has been redesigned from scratch and stores a trace of the execution in a logfile.
This trace consists of information how many bytes of input have been read so far,
and the symbolic constraints of each branching point with flag whether the branch had been taken or not, in SMT-LIB syntax.

\section{Discussion: Strengths and Weaknesses}
\label{sec:discussion}

\section{Tool Setup and Configuration}
\label{sec:project}


\paragraph{Configuration.}

\paragraph{Participation.}
\Legion/\SymCC participates in all categories of \code{Cover-Branches}.

\paragraph{Contributors \& Acknowledgement.}

\Legion/\SymCC is developed and maintained by the authors.
The source code is available via \url{https://github.com/gernst/legion-symcc} under the MIT license.
The original \Legion implementation is available at \url{https://github.com/Alan32Liu/Legion} under the MIT license.

\bibliographystyle{splncs04}
\bibliography{korn.bib}

\end{document}
